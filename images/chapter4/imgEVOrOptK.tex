%====================================================================================%
%                           Universidade Federal Fluminense                          %
%                          Mestrado em Ciência da Computação                         %
%------------------------------------------------------------------------------------%
% Título da Dissertação:                                                             %
%    Um algoritmo heurístico híbrido para o problema de roteamento de veículos com   %
%    coleta e entrega simultânea                                                     %
%------------------------------------------------------------------------------------%
% Autor      : Marcio Tadayuki Mine                           <mmine@ic.uff.br>      %
% Orientador : Luiz Satoru Ochi                               <satoru@ic.uff.br>     %
% Orientador : Marcone Jamilson Freitas Souza                 <marcone@iceb.ufop.br> %
% ADAPTADO POR: Vitor Nazario Coelho    - Maio de 2013        <vncoelho@gmail.combr> %
%====================================================================================%

\begin{figure}[http]
	\begin{center}
		\begin{tikzpicture}[scale=.086]

			\node[depot] (n0)  at (35,35) {};
			\node[city]  (n1)  at (49,41) {};
			\node[city]  (n2)  at (17,35) {};
			\node[city]  (n3)  at (45,55) {};
			\node[city]  (n4)  at (21,53) {};
			\node[city]  (n5)  at (60,55) {};
			\node[city]  (n6)  at (60,30) {};
			\node[city]  (n7)  at (35,50) {};
			\node[city]  (n8)  at (25,22) {};
			\node[city]  (n9)  at (5,30)  {};
			\node[city]  (n10) at (40,20) {};
			\node[city]  (n11) at (65,45) {};
			\node[city]  (n12) at (10,45) {};
			\node[city]  (n13) at (5,55)  {};
			\node[city]  (n14) at (33,64) {};
			\node[city]  (n15) at (20,65) {};
			\node[city]  (n16) at (69,35) {};
			\node[city]  (n17) at (55,65) {};
			\node[city]  (n18) at (65,63) {};
			\node[city]  (n19) at (12,24) {};

			\node[depotFont]   at (n0)    {0};
			\node[cityFont]    at (n1)    {4};
			\node[cityFont]    at (n2)    {13};
			\node[cityFont]    at (n3)    {3};
			\node[cityFont]    at (n4)    {1};
			\node[cityFont]    at (n5)    {5};
			\node[cityFont]    at (n6)    {6};
			\node[cityFont]    at (n7)    {7};
			\node[cityFont]    at (n8)    {11};
			\node[cityFont]    at (n9)    {9};
			\node[cityFont]    at (n10)   {10};
			\node[cityFont]    at (n11)   {8};
			\node[cityFont]    at (n12)   {12};
			\node[cityFont]    at (n13)   {2};
			\node[cityFont]    at (n14)   {19};
			\node[cityFont]    at (n15)   {15};
			\node[cityFont]    at (n16)   {16};
			\node[cityFont]    at (n17)   {14};
			\node[cityFont]    at (n18)   {18};
			\node[cityFont]    at (n19)   {17};

			\path[->, olive] (n0)  edge (n7)
					  (n7)  edge (n14)
					  (n14) edge (n15)
					  (n15) edge (n4)
					  (n4)  edge (n13)
					  (n13) edge (n12)
					  (n12) edge (n0);

			\path[->,teal] (n0)  edge (n10)
					  (n10) edge (n8)
					  (n8)  edge (n19)
					  (n19) edge (n9)
					  (n9)  edge (n2)
					  (n2)  edge (n0);

			\path[->] (n0)  edge (n1)
					  (n1)  edge (n6)
					  (n6)  edge (n16)
					  (n16) edge (n11);
			\path[->,dashed]		  (n11) edge (n5)
					  (n5)  edge (n18);
			\path[->]		  (n18) edge (n17)
					  (n17) edge (n3);
			\path[->,dashed]		  (n3)  edge (n0);

			\node[rectangle, scale=.8, olive] at (25,47) {Route 1};
			\node[rectangle, scale=.8] at (53,47) {Route 2};
			\node[rectangle, scale=.8, teal] at (25,29) {Route 3};

			\node[depot] (m0)  at (115,35) {};
			\node[city]  (m1)  at (129,41) {};
			\node[city]  (m2)  at (97,35)  {};
			\node[city]  (m3)  at (125,55) {};
			\node[city]  (m4)  at (101,53) {};
			\node[city]  (m5)  at (140,55) {};
			\node[city]  (m6)  at (140,30) {};
			\node[city]  (m7)  at (115,50) {};
			\node[city]  (m8)  at (105,22) {};
			\node[city]  (m9)  at (85,30)  {};
			\node[city]  (m10) at (120,20) {};
			\node[city]  (m11) at (145,45) {};
			\node[city]  (m12) at (90,45)  {};
			\node[city]  (m13) at (85,55)  {};
			\node[city]  (m14) at (113,64) {};
			\node[city]  (m15) at (100,65) {};
			\node[city]  (m16) at (149,35) {};
			\node[city]  (m17) at (135,65) {};
			\node[city]  (m18) at (145,63) {};
			\node[city]  (m19) at (92,24)  {};

			\node[depotFont]   at (m0)    {0};
			\node[cityFont]    at (m1)    {4};
			\node[cityFont]    at (m2)    {13};
			\node[cityFont]    at (m3)    {3};
			\node[cityFont]    at (m4)    {1};
			\node[cityFont]    at (m5)    {5};
			\node[cityFont]    at (m6)    {6};
			\node[cityFont]    at (m7)    {7};
			\node[cityFont]    at (m8)    {11};
			\node[cityFont]    at (m9)    {9};
			\node[cityFont]    at (m10)   {10};
			\node[cityFont]    at (m11)   {8};
			\node[cityFont]    at (m12)   {12};
			\node[cityFont]    at (m13)   {2};
			\node[cityFont]    at (m14)   {19};
			\node[cityFont]    at (m15)   {15};
			\node[cityFont]    at (m16)   {16};
			\node[cityFont]    at (m17)   {14};
			\node[cityFont]    at (m18)   {18};
			\node[cityFont]    at (m19)   {17};

			\path[->, olive] (m0)  edge (m7)
					  (m7)  edge (m14)
					  (m14) edge (m15)
					  (m15) edge (m4)
					  (m4)  edge (m13)
					  (m13) edge (m12)
					  (m12) edge (m0);

			\path[->,teal] (m0)  edge (m10)
					  (m10) edge (m8)
					  (m8)  edge (m19)
					  (m19) edge (m9)
					  (m9)  edge (m2)
					  (m2)  edge (m0);

			\path[->] (m0)  edge (m1)
					  (m1)  edge (m6)
					  (m6)  edge (m16)
					  (m16) edge (m11);
			\path[->,very thick]		  (m11) edge (m18);
			\path[->]		  (m18) edge (m17)
					  (m17) edge (m3);
			\path[->,very thick]		  (m3)  edge (m5)
					  (m5)  edge (m0);

			\node[rectangle, scale=.8, olive] at (105,47) {Route 1};
			\node[rectangle, scale=.8] at (136,45) {Route 2};
			\node[rectangle, scale=.8, teal] at (105,29) {Route 3};

			% Legenda:
			\draw (40,15) -- (110,15);
			\draw (40,9) -- (110,9);

			\node[depot, scale=.8] at (45,12) {};
			\node[rectangle, scale=.8] at (57,12) {Depot};

			\node[city, scale=.85] at (70,12) {};
			\node[rectangle, scale=.8] at (81,12) {Client};

			\draw[->] (91,12) -- (94,12);
			\node[rectangle, scale=.8] at (102,12) {Route};
		\end{tikzpicture}
		\caption{Example of \textit{Or-optk} Move}\label{fig:EVOrOptK}
	\end{center}
\end{figure}
