%====================================================================================%
%                           Universidade Federal Fluminense                          %
%                          Mestrado em Ciência da Computação                         %
%------------------------------------------------------------------------------------%
% Título da Dissertação:                                                             %
%    Um algoritmo heurístico híbrido para o problema de roteamento de veículos com   %
%    coleta e entrega simultânea                                                     %
%------------------------------------------------------------------------------------%
% Autor      : Marcio Tadayuki Mine                           <mmine@ic.uff.br>      %
% Orientador : Luiz Satoru Ochi                               <satoru@ic.uff.br>     %
% Orientador : Marcone Jamilson Freitas Souza                 <marcone@iceb.ufop.br> %
%====================================================================================%

\begin{figure}[H]
	\begin{center}
		\begin{tikzpicture}[scale=.12]

			\node[depot] (n0)  at (35,35) {};
			\node[city]  (n1)  at (49,41) {};
			\node[city]  (n2)  at (17,35) {};
			\node[city]  (n3)  at (45,55) {};
			\node[city]  (n4)  at (21,53) {};
			\node[city]  (n5)  at (60,55) {};
			\node[city]  (n6)  at (60,30) {};
			\node[city]  (n7)  at (35,50) {};
			\node[city]  (n8)  at (25,22) {};
			\node[city]  (n9)  at (5,30)  {};
			\node[city]  (n10) at (40,20) {};
			\node[city]  (n11) at (65,45) {};
			\node[city]  (n12) at (10,45) {};
			\node[city]  (n13) at (5,55)  {};
			\node[city]  (n14) at (33,64) {};
			\node[city]  (n15) at (20,65) {};
			\node[city]  (n16) at (69,35) {};
			\node[city]  (n17) at (55,65) {};
			\node[city]  (n18) at (65,63) {};
			\node[city]  (n19) at (12,24) {};

			\node[depotFont]   at (n0)    {0};
			\node[cityFont]    at (n1)    {1};
			\node[cityFont]    at (n2)    {2};
			\node[cityFont]    at (n3)    {3};
			\node[cityFont]    at (n4)    {4};
			\node[cityFont]    at (n5)    {5};
			\node[cityFont]    at (n6)    {6};
			\node[cityFont]    at (n7)    {7};
			\node[cityFont]    at (n8)    {8};
			\node[cityFont]    at (n9)    {9};
			\node[cityFont]    at (n10)   {10};
			\node[cityFont]    at (n11)   {11};
			\node[cityFont]    at (n12)   {12};
			\node[cityFont]    at (n13)   {13};
			\node[cityFont]    at (n14)   {14};
			\node[cityFont]    at (n15)   {15};
			\node[cityFont]    at (n16)   {16};
			\node[cityFont]    at (n17)   {17};
			\node[cityFont]    at (n18)   {18};
			\node[cityFont]    at (n19)   {19};

			\path[->] (n0)  edge (n7)
					  (n7)  edge (n14)
					  (n14) edge (n15)
					  (n15) edge (n4)
					  (n4)  edge (n13)
					  (n13) edge (n12)
					  (n12) edge (n0);

			\path[->] (n0)  edge (n10)
					  (n10) edge (n8)
					  (n8)  edge (n19)
					  (n19) edge (n9)
					  (n9)  edge (n2)
					  (n2)  edge (n0);

			\path[->] (n0)  edge (n1)
					  (n1)  edge (n6)
					  (n6)  edge (n16)
					  (n16) edge (n11)
					  (n11) edge (n5)
					  (n5)  edge (n18)
					  (n18) edge (n17)
					  (n17) edge (n3)
					  (n3)  edge (n0);

			\draw (12,14) -- (66,14);
			\draw (12,10) -- (66,10);
			
			\node[depot, scale=.75] at (17,12) {};
			\node[rectangle, scale=.75] at (25,12) {Depósito};

			\node[city, scale=.8] at (37,12) {};
			\node[rectangle, scale=.75] at (44,12) {Cliente};
			
			\draw[->] (51,12) -- (56,12);
			\node[rectangle, scale=.75] at (60,12) {Rota};
			
		\end{tikzpicture}
		\caption{Exemplo de uma solução do PRVCES.}\label{fig:SolucaoPRVCES}
	\end{center}
\end{figure}
